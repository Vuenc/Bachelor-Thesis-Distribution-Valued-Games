% !TeX spellcheck = en_US
% Dirty hack to disable recompiling the bibliography every time, reducing compilation times. Comment out to re-compile bibliography.
% !TeX TXS-program:recompile-bibliography = donothing

% Use biber as bibliography tool
% !TeX TXS-program:bibliography = txs:///biber

\documentclass[a4paper]{scrreprt}

\usepackage{amsfonts} 
\usepackage{amsmath}
\usepackage{amsthm} %Proof Environment
\usepackage{amssymb}
\usepackage{marvosym}
\usepackage{stmaryrd} %\lightning
\usepackage{latexsym} %\leadsto

\usepackage[utf8]{inputenc}
\usepackage[T1]{fontenc}
\usepackage[english]{babel} 

\usepackage{pdfpages} % include pdf pages
\usepackage{hyperref}
\usepackage{graphicx}

\usepackage{enumitem}
\usepackage{multicol}
\usepackage{chngcntr}

\usepackage{xifthen}

\usepackage{listings} % source listings
\usepackage{xcolor} % defining own colors: \color

\usepackage{dsfont} % \1 for identity functions

\usepackage{etoolbox}

% custom math commands:

% column vectors
\newcount\xveccount
\renewcommand*\vec[1]{
	\global\xveccount#1
	\begin{pmatrix}
		\vecnext
	}
	\def\vecnext#1{
		#1
		\global\advance\xveccount-1
		\ifnum\xveccount>0
		\\
		\expandafter\vecnext
		\else
	\end{pmatrix}
	\fi
}  
\newcommand{\vectwo}[3][0pt]{\begin{pmatrix}#2\\[#1] #3\end{pmatrix}}
\newcommand{\vecthree}[4][0pt]{\begin{pmatrix}#2\\[#1] #3\\[#1] #4\end{pmatrix}}
\newcommand{\vecfour}[5][0pt]{\begin{pmatrix}#2\\[#1] #3\\[#1] #4\\[#1] #5\end{pmatrix}}
\newcommand{\vecfive}[6][0pt]{\begin{pmatrix}#2\\[#1] #3\\[#1] #4\\[#1] #5\\[#1] #6\end{pmatrix}}

% * as multiplication dot
\mathcode`\*="8000
{\catcode`\*\active\gdef*{\cdot}}

% common number set symbols
\newcommand{\N}{\mathbb{N}}
\newcommand{\Z}{\mathbb{Z}}
\newcommand{\Q}{\mathbb{Q}}
\newcommand{\R}{\mathbb{R}}
\newcommand{\C}{\mathbb{C}}

% useful mathematical operators
\DeclareMathOperator{\Pot}{\mathcal{P}}
\DeclareMathOperator{\BigO}{\mathcal{O}}
\DeclareMathOperator{\inv}{^{-1}}
\renewcommand{\Re}[1]{\text{Re}\left( #1 \right)}
\renewcommand{\Im}[1]{\text{Im}\left( #1 \right)}
\newcommand{\abs}[1]{\left| #1 \right|}
\DeclareMathOperator{\vspan}{\text{span}}
\newcommand{\setspan}[1]{\vspan\left(\left\{#1\right\}\right)}
\DeclareMathOperator{\rang}{\text{rang}}
\newcommand{\integral}[4]{\int_{#1}^{#2} #3 \, d#4}
\newcommand{\lintegral}[3]{\int_{#1} #2 \, d#3}
\newcommand{\comp}{^\complement}
\newcommand{\ra}{\rightarrow}
\newcommand{\lra}{\Leftrightarrow} % equivalence arrow
\newcommand{\1}[1][]{\mathds{1}\ifthenelse{\isempty{#1}}{}{_{#1}}}
\newcommand{\dummydot}{\,\cdot\,}
\DeclareMathOperator{\im}{\text{im}}
\newcommand{\innerprod}[2]{\left\langle #1, #2 \right\rangle}
\DeclareMathOperator{\supp}{supp}
\newcommand{\transposed}{^{T}}
\DeclareMathOperator{\argmax}{argmax}
\DeclareMathOperator{\argmin}{argmin}
\newcommand{\colonlra}{\mathrel{\vcentcolon\Leftrightarrow}} % properly typeset :<=>

% switch-case structure
\newcommand{\ifequals}[4]{\ifthenelse{\equal{#1}{#2}}{#3}{#4}}
\newcommand{\case}{} % Dummy, so \renewcommand has something to overwrite...
\newenvironment{switch}[1]{\renewcommand{\case}[3]{\ifequals{#1}{##1}{##2}{##3}}}{} %Usage: \begin{switch}{value}, \case{value}{then}{else}

% shortcut for inline pmatrix
\newcommand{\pmat}[1]{\begin{pmatrix}#1\end{pmatrix}}

% brutally enforce centering the contents
\newcommand{\centerbrutally}[1]
{
	\centerline{
		\begin{minipage}{\linewidth}
			#1
	\end{minipage}}
}

\definecolor{verylightgray}{gray}{0.97}
\definecolor{purple}{RGB}{127,0,116} % used for keyword coloring in source code listings
\definecolor{brickred}{rgb}{0.56, 0.175, 0.231} % used for string coloring in source code listing

% norm ||x||
\newcommand{\norm}[1]{\left\lVert#1\right\rVert}

% centered inline graphics
\newcommand{\includegraphicsinline}[2][]{\begingroup\setbox0=\hbox{\includegraphics[#1]{#2}}\parbox{\wd0}{\box0}\endgroup}

% tilde in lstlisting
\lstset{literate=%
    {~}{{\textasciitilde}}1
}

% shortcuts: for limits/series going to infinity
\newcommand{\liminfty}[1]{\lim\limits_{#1 \rightarrow \infty}}
\newcommand{\ser}[1]{\sum\limits_{\ifthenelse{\isin{=}{#1}}{#1}{#1=1}}^{\infty}}
\newcommand{\toinfty}[1]{\xrightarrow{\;#1 \to \infty\;}}

% shortcut for multiple cases in math formulas: '"1", falls "2", "3", falls "4"' (bzw. 'sonst', falls leer)
\newcommand{\twocases}[4]
{
    \begin{cases}
        #1, &\text{if } #2 \\
        #3, &\ifthenelse{\isempty{#4}}{\text{else}}{\text{if } #4}
    \end{cases}
}

% shortcut for parentheses and sets
\newcommand{\pars}[1]{\left(#1\right)}
\newcommand{\bigpars}[1]{\bigl(#1\bigr)}
\newcommand{\biggpars}[1]{\biggl(#1\biggr)}
\newcommand{\set}[1]{\{#1\}}
\newcommand{\bigset}[1]{\bigl\{#1\bigr\}}
\newcommand{\biggset}[1]{\biggl\{#1\biggr\}}
\newcommand{\bigmid}{\bigm\vert}
\newcommand{\biggmid}{\biggm\vert}

\newenvironment{correction}[1]{\color{red}\textbf{\underline{#1}}\\}{\ignorespacesafterend}

\makeatletter
\patchcmd{\upbracefill}{\m@th}{\scriptscriptstyle\m@th}{}{}
\patchcmd{\upbracefill}{$\braceld$}{$\scriptstyle\braceld$}{}{}
\patchcmd{\upbracefill}{\bracelu}{\bracelu\mkern-1mu}{}{}
\patchcmd{\upbracefill}{\hfill\braceru}{\hfill\mkern-1mu\braceru}{}{}
\makeatother

\newcommand{\smallmat}[1]{\left(\begin{smallmatrix}#1\end{smallmatrix}\right)}
\newcommand{\undersetbrace}[2]{\underset{#2}{\underbrace{#1}}}

% theorem-like environments

\theoremstyle{definition}
\newtheorem{thm}{Theorem}[chapter] % reset theorem numbering for each chapter
\newtheorem*{thm*}{Theorem}
\newtheorem{defn}[thm]{Definition} % definition numbers are dependent on theorem numbers
\newtheorem*{defn*}{Definition}
\newtheorem{ex}[thm]{Example} % same for example numbers
\newtheorem*{ex*}{Example}
\newtheorem{lemma}[thm]{Lemma} % same for Lemma numbers
\newtheorem*{lemma*}{Lemma}
\newtheorem{cor}[thm]{Corollary}
\newtheorem*{cor*}{Corollary}
\newtheorem{comm}[thm]{Comment}
\newtheorem*{comm*}{Comment}
\newtheorem{notation}[thm]{Notation}
\newtheorem*{notation*}{Notation}


\usepackage[citestyle=alphabetic,bibstyle=alphabetic]{biblatex}
\usepackage{csquotes}

\usepackage{setspace}
\usepackage[headsepline]{scrlayer-scrpage}

\usepackage{mathtools} % \bigtimes

% no ident, combined with reasonable paragraph spacing
\onehalfspacing
\setlength{\parindent}{0em}
\setlength{\parskip}{1.7ex}

\addbibresource{../Bachelor-Thesis.bib}

\let\epsilon\varepsilon
\DeclareMathOperator{\A}{\mathcal{A}}
\DeclareMathOperator{\leqst}{\leq_{\text{st}}}
\DeclareMathOperator{\leqtail}{\leq_{\text{tail}}}
\DeclareMathOperator{\geqtail}{\geq_{\text{tail}}}
\DeclareMathOperator{\F}{\mathcal{F}}
\DeclareMathOperator{\RVs}{\mathcal{D}}
\DeclareMathOperator{\B}{\mathcal{B}}
\DeclareMathOperator{\supp}{supp}
\DeclareMathOperator{\leqlex}{\leq_{\textsc{Lex}}}

% Style for displaying source code listings
\lstset{showspaces=false,
    keywordstyle=\ttfamily\bfseries\color{purple},
    basicstyle=\ttfamily,
    numbers=left,
    numberstyle=\tiny,
    commentstyle=\color{gray},
    breaklines=true,
    postbreak=\mbox{\textcolor{lightgray}{$\hookrightarrow$}\space},
    showstringspaces=false,
    stringstyle=\color{brickred}
}

%\setlist[enumerate,1]{label=(\roman*)}

\ihead{Vincent Bürgin}
\ohead{[Work in Progress]}

\begin{document}
    
    \setcounter{chapter}{4}
    
%    \chapter{Games with Random Variables as Payoffs}
%    In this chapter, we define games in the game-theoretic sense, where the payoffs are no longer real numbers, but rather real-valued random variables. The description follows the approach found in \cite{bib:rassGameRiskManagI,bib:rassGameRiskManagII,bib:rassTotalOrderingOnLossDistributions}. % TODO but deviates!

    \section{The Stochastic Tail Order}
    
    We consider real-valued random variables on some probability space $(\Omega, \A, P)$. In order to define a total order on a subset of the real-valued random variables, we make some assumptions about random variables which can be ordered:
    % TODO write a proper introduction sentence
    
    \begin{defn}
        We call a real-valued random variable $X$ is \emph{tail-orderable} if 
        it only takes values greater than 1, and
         one of the following holds:
        \begin{enumerate}
            \item $X$ is discrete with finite support.
            \item $X$ is absolute continuous, and its density $f$ is zero outside a compact interval $[a_1, b_1]$, as well as continuous von $(a_1, b_1)$.
        \end{enumerate} 
%    either discrete with finite support, or absolutely continuous with its density $f$ being zero outside a compact interval $[a_1, b_1]$, and continuous on $(a_1, b_1)$. 
        The set of tail-orderable random variables over $(\Omega, \A, P)$ is denoted by $\F$.
    \end{defn}

    
    From now on, we assume that $X, Y$ are real-valued random variables either both discrete or both absolutely continuous, with densities/probability mass functions $f$ and $g$, respectively. % TODO ok if one is discrete, the other AC? relevant to \leqtail being a total ordering on \F...
    
    
    
    \begin{defn}[Stochastic Tail Order]
        Let $X, Y$ be tail-orderable. We define the \emph{stochastic tail order} $\leqtail$ by
        \footnote{\cite{bib:rassTotalOrderingOnLossDistributions,bib:rassGameRiskManagI} use the symbol $\preceq$ for the stochastic tail order. Since we use different stochastic orders in this text, for clarity we instead use $\leqtail$.}
        \[ X \leqtail Y :\lra \exists K \in \N: \forall k \geq K: E(X^k) \leq E(Y^k)\]
        \label{def:tailorder}
    \end{defn}
    
    \begin{lemma}
        $\leqtail$ from definition \ref{def:tailorder} is a transitive and total binary relation on $\F$, i.e. a total preorder.
    \end{lemma}
    \begin{proof}
        % TODO
    \end{proof}
    
    \subsection{Characterization of the Stochastic Tail Order}
    We want to give a better intuition for what the stochastic tail order $\leqtail$ means.
    \cite{bib:shakedStochasticOrders} defines a similar ordering $\leqst$, called the \emph{(usual) stochastic order}, which is defined as
    \footnote{\cite{bib:shakedStochasticOrders} give a slightly different, equivalent definition using plain probabilities instead of distribution functions.}
    \begin{gather}
        X \leqst Y \lra \forall x \in \R: F_X(x) \geq F_Y(y)
    \end{gather}
    In other words, $X \leqst Y$ if $X$'s distribution is point-wise greater than $Y$'s, so in a certain sense $X$ takes on smaller values with higher probability.
    
    % TODO rethink, -formulate this whole thing. Overtake is somewhat the wrong term I think? But not sure? Try some examples
    We will see that $X \leqtail Y$ often can be characterized in a similar, but less strict manner: for $X \leqst Y$ to hold, $F_X$ does not need to be point-wise greater than $F_Y$ everywhere, but only \emph{eventually}, i.e. from a certain point on. The interpretation of this is that from a pessimistic risk-assessment point-of-view, $X$ should be preferred if it reduces the probability of the \emph{highest losses}, no matter what happens in the regions further to the left where the losses are smaller. This can be formulated as a sufficient condition, but not an equivalence: there are pathological counter-examples where even on a bounded interval, no distribution function eventually overtakes the other, but both functions cross over infinitely often.
    % TODO present such an counter-example
    
    
    \begin{lemma}[Sufficient condition for $\leqtail$]
        Let $X, Y$ be tail-orderable.
%        either both absolutely continuous or both discrete with finite support, with corresponding probability mass or density functions $f, g$, respectively. Suppose that
%        \begin{gather*}
%            \exists x_0: f(x_0) < g(x_0), \forall x \geq x_0: f(x) \leq g(x)
%        \end{gather*}
%        Then $X \leqtail Y$. % TODO strict?
        \begin{enumerate}
            \item If $X, Y$ are absolutely continuous with densities $f, g$: If there is some $x_0 \in \R$ and a non-degenerate interval $[x_0, x_0 + \delta]$ such that
            % TODO technically, non-degenerate interval
            \begin{gather*}
                \forall x \in [x_0, x_0 + \delta]: f(x) < g(x) ~~ \wedge ~~ \forall x \geq x_0: f(x) \leq g(x),
            \end{gather*}
            then $X \leqtail Y$. % TODO strict?
            \label{item:leqtailSuffCondAC}
            
            
            \item If $X, Y$ are discrete with finite support and probability mass functions $f, g$: If there is some $x_0 \in \R$ such that
            \begin{gather*}
            f(x_0) < g(x_0) ~~ \wedge ~~ \forall x \geq x_0: f(x) \leq g(x),
            \end{gather*}
            then $X \leqtail Y$. % TODO strict?
            \label{item:leqtailSuffCondDiscrete}
        \end{enumerate}
    \end{lemma}
    \begin{proof}
        \newcommand{\mkx}{m_k^X}
        \newcommand{\mky}{m_k^Y}
        We denote the $k$'th moments as $\mkx = E(X^k)$, $\mky = E(Y^k)$, and want to show that $\mkx \leq \mky$ eventually.
        As $X, Y$ have bounded supports, let $a, b$ 
        such that $X \in [a, b], Y \in [a, b]$ almost surely.
        
        On case \ref{item:leqtailSuffCondAC}: Suppose $X, Y$ are absolutely continuous. Then
        \begin{multline*}
            \mky - \mkx 
            = \lintegral{\R}{t^k g(t)}{\lambda(t)} - \lintegral{\R}{t^k f(t)}{\lambda(t)}
            = \lintegral{[a, b]}{t^k(g(t) - f(t))}{\lambda(t)} \\
            = \lintegral{[a, x_0]}{t^k (g(t) - f(t))}{\lambda(t)} + \lintegral{[x_0, b]}{t^k (g(t) - f(t))}{\lambda(t)} 
        \end{multline*}
        We simplify the left integral: we use that $g - f \geq -1$ since both are densities, and $\forall t \in [a, x_0]: t^k \leq x_0^k$, hence $t^k*(g(t) - f(t)) > -x_0^k$:
        % TODO this whole approximation is STILL a bit fishy.... sure it's right? (but possibly yes)
        \begin{gather*}
             \lintegral{[a, x_0]}{t^k (g(t) - f(t))}{\lambda(t)} 
             \geq x_0^k*\lintegral{[a, x_0]}{(-1)}{\lambda(t)} 
             = - x_0^k*(x_0-a) 
             =: - x_0 * c_{\text{left}}
        \end{gather*}
        For the right integral, we use that $(g - f) > 0$ on $[x_0, x_0 + \delta]$. Let $x_1 := x_0 + \frac{\delta}{2}$.
        We first identify a subset of $I := [x_1, x_0 + \delta]$ with positive measure where $(g-f)$ is bounded away from zero:
        Let $A_n := \set{x \in I \mid g(x) - f(x) > \frac{1}{n}}$.
        % TODO and these sets are Borel
        Then $I = \bigcup_n A_n$ and by continuity of measures, $\lambda(A_n) \toinfty{n} \lambda(I) > 0$. This way we could construct some $A \subseteq I$, $\epsilon$ for which $\lambda(A) > 0$ and $\forall x \in A: g(x) - f(x) \geq \epsilon$. 
        Then
        \begin{multline*}
            \lintegral{[x_0, b]}{t^k (g(t) - f(t))}{\lambda(t)} 
            \geq \lintegral{A}{t^k (g(t) - f(t))}{\lambda(t)} \\
            \geq \lintegral{A}{t^k * \epsilon}{\lambda(t)}
            \geq x_1^k * (\epsilon * \lambda(A_l))
            =: x_1^k * c_{\text{right}}
        \end{multline*}
        
        Further,
        \begin{gather*}
            \mky - \mkx
            \geq s_1^k * c_{\text{right}} - x_0^k * c_{\text{left}}
        \end{gather*}
        So since both expressions grow exponentially with $k$, and $s_1 > x_0$, it follows that for $k$ large enough, $\mky - \mkx > 0$ and hence $\mky > \mkx$.
        
        
        On case \ref{item:leqtailSuffCondDiscrete}: Suppose that $X, Y$ are discrete with finite support. Let 
        \begin{gather*}
            S = \supp(X) \cup \supp(Y) = \set{x_1, \dots, x_n}, \text{ where } x_1 \leq \dots \leq x_n
        \end{gather*}
        
        Let $m$ be the maximal index where $f(x_m) < g(x_m)$, i.e. $f(x_i) = g(x_i) \;\forall m < i \leq n$.
        Then
        \begin{multline*}
            \mky - \mkx
            = \biggpars{\sum_{i=1}^{n} x_i^k \,g(x_i)} - \biggpars{\sum_{i=1}^{n} x_i^k \,f(x_i)}
            = \biggpars{\sum_{i=1}^{m} x_i^k \,(g(x_i) - f(x_i))} \\
            = \biggpars{\sum_{i=1}^{m-1} x_i^k \,(g(x_i) - f(x_i))} + x_m^k \,(g(x_m) - f(x_m))
        \end{multline*}
        
        Similar to above, we use $(g -f) \geq -1$ and $x_i \leq x_{m-1}$ for all $i < m$ to obtain a bound on the sum on the left:
        \begin{gather*}
            \biggpars{\sum_{i=1}^{m-1} x_i^k \,(g(x_i) - f(x_i))}
            \geq x_{m-1}^k * ((m-1) * (-1))
            =: -x_{m-1}^k * c_{\text{left}}
        \end{gather*}
        We set $c_{\text{right}} := (g(x_m) - f(x_m))$. Then
        \begin{gather*}
            \mky - \mkx \geq x_m^k * c_{\text{right}} -x_{m-1}^k * c_{\text{left}} > 0 \text{ for $k$ large enough.}
        \end{gather*}        
    \end{proof}
    
    
    
    
    
    
    
%    \begin{lemma}
%        Let $X, Y$ be tail-orderable.
%        Suppose there exists some $x_0 \in \R$ such that $F_Y(x_0) < 1$, and $\forall x \geq x_0: F_X(x) \geq F_Y(x)$:
%        Then $X \leqtail Y$, i.e. $Y$'s moment sequence eventually overtakes $X$'s.
%        % TODO not actually overtake, but can also stay the same
%        
%        
%%        The following are equivalent:
%%        \begin{enumerate}
%%            \item $X \leqtail Y$, i.e. $Y$'s moment sequence eventually overtakes $X$'s.
%%                        
%%            \item There exists some $x_0 \in \R$ such that $F_Y(x_0) \neq 1$, and $\forall x \geq x_0: F_X(x) \geq F_Y(x)$.
%%        \end{enumerate}
%    \end{lemma}
%    \begin{proof}
%        \newcommand{\mkx}{m_k^X}
%        \newcommand{\mky}{m_k^Y}
%        We denote the $k$'th moments as $\mkx = E(X^k)$, $\mky = E(Y^k)$, and want to show that $\mkx \leq \mky$ eventually.
%        As $X, Y$ have bounded supports, there exists some $l > x_0$ where $F_X(l) = F_Y(l) = 1$, i.e. all probability mass lies left of $l$.
%        
%        \begin{multline*}
%            \mkx 
%            = \lintegral{\R}{x^k}{P^X(x)}
%            = \lintegral{(-\infty, x_0)}{x^k}{P^X(x)} + \lintegral{[x_0, l]}{x^k}{P^X(x)} \\
%            \leq \lintegral{(-\infty, x_0)}{x^k}{P^X(x)} + x_0^k * P(\set{X \in (-\infty, x_0)}) \\ % TODO wrong?
%            = \lintegral{(-\infty, x_0)}{x^k}{P^X(x)} + x_0^k * F_X(x_0)
%        \end{multline*}
%        
%        Now pick a $s \in (x_0, l)$ s.t. $F_y(s) < 1$.
%        % TODO prove that s exists?
%        Then $s > x_0$, and $P(\set{Y > s}) > 0$. Similar to above, we get:
%        \begin{multline*}
%            \mky = \lintegral{(-\infty, x_0)}{y^k}{P^Y(y)} + \lintegral{[x_0, l]}{y^k}{P^Y(y)} \\
%            \geq \lintegral{(-\infty, x_0)}{y^k}{P^Y(y)} + \lintegral{[s, l]}{y^k}{P^Y(y)} 
%            \geq \lintegral{(-\infty, x_0)}{y^k}{P^Y(y)} + s^k*P(\set{Y \in [s, l]}) \\
%            = \lintegral{(-\infty, x_0)}{y^k}{P^Y(y)} + s^k*(1 - F_Y(s))
%        \end{multline*}
%        
%        % TODO prove!
%    \end{proof}
%
%
%    The same condition can also be applied to the density or probability mass functions: we will see that $X \leqtail Y$ if $X$'s \emph{vector of probability masses} on the common support is smaller \emph{lexicographically from the right} than $Y$'s, and similarly for density functions.
%
%    \begin{cor}~
%        \begin{enumerate}
%            \item If both $X$ and $Y$ are discrete with finite support, with probability mass functions $f, g$, and $S = \supp(X) \cup \supp(Y) = \set{x_1, \dots, x_n}$ with $x_1 \leq \dots \leq x_n$: Then
%            \begin{alignat}{2}
%                 &&&~ (f(x_n), \dots, f(x_1)) \;\leqlex\; (g(x_n), \dots, g(x_1)) \\
%                 &\Longleftrightarrow&&~ (f = g) \vee (\exists k: f(x_k) < g(x_k) \wedge \forall j > k: f(x_j) = g(x_j)) \\
%                 & \Longleftrightarrow&&~ X \leqtail Y            
%            \end{alignat}
%            \label{item:leqtailDiscreteLex}
%            
%            \item If both $X$ and $Y$ are absolute continuous with probability density functions, then
%            \begin{gather}
%                (f = g) \vee (\exists x_0 \in \R: f(x_0) < g(x_0) \wedge \forall x \geq x_0: f(x) \leq g(x)) ~\Longrightarrow~ X \leqtail Y
%            \end{gather}
%            \label{item:leqtailACLex}
%            
%            % TODO check the conditions again. The AC case uses \leq where the discrete case uses =
%        \end{enumerate}
%    \end{cor}
%    \begin{proof}
%        % TODO prove!
%    \end{proof}
    
    
    %    \nocite{*}
    %    \printbibliography
\end{document}